\subsection{Glossary}\label{sec:glossary}

\begin{itemize}
    
    \item \textbf{Batch job:} A computer program (or set of programs) scheduled to run at a time where resources are easily available, e.g. a bank applying interest to accounts at night when few customers are using their web services.
    
    \item \textbf{CSS:} \textbf{C}ascasing \textbf{S}tyle \textbf{S}heets, a language used to set how a document (written in a markup language like HTML) is presented.
    
    \item \textbf{Database:} A storage place for data. In this context, the database is an IBM mainframe situated in Dallas, which was used to store table data.
    
    \item \textbf{DataStudio:} A program by IBM used to access and modify DB2 databases.

    \item \textbf{DB2:} Relational database system made by IBM.
    
    \item \textbf{HTML:} \textbf{H}yper\textbf{t}ext \textbf{M}arkup \textbf{L}anguage. The most popular markup language used for creating web pages and web applications.
    
    \item \textbf{HTML form:} Type of web form. Web forms allows user to enter data using checkboxes, radio buttons, or text fields. The data is then sent to a server and processed by a web application.
    
    \item \textbf{JavaScript:} A programming language commonly used in web-applications. Not to be confused with the programming language Java.
    
    \item \textbf{JSP:} \textbf{J}ava\textbf{S}erver \textbf{P}ages - a language which provides an easy way to integrate Java code into HTML files. Has the file extension \texttt{.jsp}, and compiles into Java servlets. In this context, they were used to generate servlets for the different pages in the application.
    
    \item \textbf{Mainframe:} A large computer that supports many workstations, and can run several partitions of operating systems. Focuses on robustness and up-time.
    
    \item \textbf{Netbank:} Short for inter\textbf{net} \textbf{bank}, a synonym for online bank.
    
    \item \textbf{PK:} Short for \textbf{P}rimary \textbf{K}ey.
    
    \item \textbf{RDBMS:} \textbf{R}elational \textbf{D}ata\textbf{b}ase \textbf{M}anagement \textbf{S}ystem, a database management system based on the relational model invented by IBM researcher E. F. Codd in the 1970s.

    \item \textbf{Servlet:} A type of Java program commonly used to integrate Java code into web-applications. The servlet itself has the file extension \texttt{.java}, but a servlet program can also be generated from JSP code, which has the file extension \texttt{.jsp}. 
    
    \item \textbf{SQL:} \textbf{S}tructured \textbf{Q}uery \textbf{L}anguage - the language used to send commands to the database tables in order to edit either the tables themselves or the data in them.
    
    \item \textbf{Sysplex:} Short for \textbf{sys}tem com\textbf{plex}. A sysplex is a cluster of IBM mainframes (or partitions on mainframes) acting as a single unit. Used to distribute workload, recover from disaster, and ensure high availability.
    
    \item \textbf{Table:} A collection of data in the database. Similar data will be stored in the same table, for example accounts are stored in an account table. Makes it easy to store and retrieve data whenever necessary.
    
    \item \textbf{Websphere Liberty:} A type of server, developed by IBM. %(IBM afsnit)
    
    \item \textbf{z/OS:} 64-bit operating system made for IBM mainframes.
        
\end{itemize}


